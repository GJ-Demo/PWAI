% Prosty przykład dokumentu LaTeXa
\documentclass{article}

% użyte pakiety
\usepackage{polski}
\usepackage[utf8]{inputenc}
\usepackage{amsmath}
\usepackage{amssymb}
\usepackage{hyperref}
%\usepackage{amsthm}

\newtheorem{theorem}{Twierdzenie}[subsection]
\newtheorem{definition}[theorem]{Definicja}

\title{Prosty przykład dokumentu}

\begin{document}

\maketitle

\section{\TeX + \LaTeX~-- wstęp}

\TeX~-- system składania tekstu technicznego.

\LaTeX~-- oprogramowanie do składania tekstu z użyciem \TeX~i rozszerzony język znaczników służących do opisywania (składania) dokumentu.

,,X'' to greckie $\chi$ (czyt. ,,chi''), zatem \textbf{Latex} czyta się \textbf{,,latech''}.

\bigskip

\textbf{Przygotowanie dokumentu:}

\begin{center}
\fbox{Źródło dokumentu Latex: plik tekstowy (typowo z rozszerzeniem .tex)}
\bigskip

	$\Downarrow$
(kompilator Latexa)
	$\Downarrow$

\bigskip	
\fbox{Gotowy dokument (np. pdf)}
\end{center}	
\bigskip

\textbf{Popularne dystrybucje \LaTeX a (kompilator + biblioteki + system dystrybucji bibliotek):}

\begin{itemize}
	\item TeX Live: \url{https://www.tug.org/texlive/}
	\item MiKTeX: \url{https://miktex.org}
	\item MacTeX, proTeXt, TinyTeX
\end{itemize}

\textbf{Popularne edytory:}

\begin{itemize}
	\item TeXstudio: \url{https://texstudio.org/}
	\item TeXworks: \url{https://tug.org/texworks/}
	\item Texmaker: \url{https://www.xm1math.net/texmaker/}
	\item VS Code + plugin
\end{itemize}

\textbf{Edytor webowy (wygodna opcja, bo nie trzeba nic instalować, wymaga rejestracji):}

\begin{itemize}
	\item Overleaf: \url{https://www.overleaf.com/}
\end{itemize}

\bigskip

Rozsądnie najpierw zainstalować \LaTeX a, a następnie edytor (pozwalając edytorowi znaleźć i ustawić \LaTeX a).

\bigskip

\section{Przykłady tekstu}
LaTeX składa tekst w dwóch trybach: zwykłym (domyślnym) i matematycznym.
\subsection{Tryb zwykły}
Surowe podstawy -- tekst jest domyślnie formatowany w paragrafach: pojedynczy koniec linii w pliku źródłowym
nie wpływa na formatowanie
ale
może
ułatwić
pisanie
długich
zdań.

Podwójny koniec linii rozpocznie nowy paragraf (który zostaje automatycznie wcięty).

Poza tym wcięcia (generalnie) nie mają wpływu na formatowanie i można ich używać do organizacji kodu.

Jeśli linijka jest zbyt długa, wtedy zostanie stosowanie złamana poprzez wyjustowanie lub złamanie słowa z użyciem myślnika.
\subsection{Tryb matematyczny}
Tryb matematyczny w danej linijce: rozwiązaniem równania \(x^2 - 2x + 1 = 0\) jest \(0\). Z kolei tryb matematyczny wycentrowany:
\[0 < x < 5 \land 2 < y < 3 \Rightarrow 2 < x+y < 8.\]
Specyficzne środowisko w którym używany jest tryb matematyczny, np. \texttt{align}/\texttt{align*} (pakiet \texttt{amsmath}):
\begin{align}
	2x + 4y + 1 & = 5x - 2y + 2,\\
	4y + 1 & = 3x - 2y + 2,\\
	6y + 1 & = 3x + 2,\\
	6y  & = 3x + 1,\\
	y  & = \frac{3x + 1}{6}.\\
\end{align}
\subsection{Przykłady konkretnych wyrażeń}
\begin{itemize}
\item Symbole (z \texttt{amssymb}) i indeksy: \(\mathbb{R}, \mathbb{N}, \emptyset, A_i, B^j, C_2^{j+1}\).

\item Szeregi:
\[ \sum_{i=1}^\infty \frac{1}{n} = \infty.\]

\item Całki:
\[ \int_0^\infty \frac{1}{x}\, dx = \infty.\]

\item Operacje na zbiorach:
\[ (A \cup B) \cap C = (A \cap C) \cup (B \cap C).\]

\item Logika:
\[ \neg p \land q \Rightarrow \neg p \lor q.\]

\item Macierze:
\[
\begin{pmatrix}
	1 & 2 \\
	3 & 4 \\
\end{pmatrix},
\begin{pmatrix}
	a_{11} & a_{12} & a_{13} \\
	a_{21} & a_{22} & a_{23} \\
\end{pmatrix}.
\]
\end{itemize}
\subsection{Środowiska}
\begin{definition}
	Funkcją ze zbioru $X$ w zbiór $Y$ nazywamy $f \subseteq X \times Y$ taki, że dla każdego $x \in X$ istnieje dokładnie jeden $y \in Y$ taki, że $\langle x, y \rangle \in f$. Notacja: $\langle x, y \rangle \in f \iff f(x) = y.$
\end{definition}

\begin{theorem}[Bolzana-Weierstrassa]
	Każdy ograniczony ciąg liczb rzeczywistych ma podciąg zbieżny.
\end{theorem}
	
\end{document}
