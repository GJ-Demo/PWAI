% Kilka elementów LaTeXa
% Jeśli nie widać spisu treści, skompiluj kod dwa razy
\documentclass{article}

\usepackage{polski}
\usepackage{amsmath}
\usepackage{amsthm}
\usepackage{amsfonts}
\usepackage{amssymb}

\usepackage{color}

\usepackage{hyperref}

\newtheorem{theorem}{Twierdzenie}[subsection]

\newcommand{\R}{\mathbb{R}}
% zastąpienie makra, np. inny symbol końca dowodu:
% \renewcommand{\qedsymbol}{\(\blacksquare\)}

\definecolor{darkgreen}{rgb}{0, 0.5, 0}

\begin{document}

\tableofcontents

\section{Pierwsza sekcja}
\section{Druga sekcja}
\subsection{Podsekcja A}
\label{sec:nazwa}
\subsubsection{Pod-podsekcja}
\subsubsection{Pod-podsekcja}
\begin{theorem}
	\label{thm:kresy}
	Każda funkcja ciągła \(f \colon [a, b] \to \R\) osiąga swoje kresy.
\end{theorem}
\begin{proof}
	Ćwiczenie dla czytelnika.
\end{proof}
\subsection{Podsekcja B}
\newpage
\section*{Sekcja bez numeru}
\section{Trzecia sekcja}

Jak widać w {\color{darkgreen}podsekcji} \ref{sec:nazwa} \ldots\\
Jak wynika z Twierdzenia \ref{thm:kresy} \ldots

\end{document}
